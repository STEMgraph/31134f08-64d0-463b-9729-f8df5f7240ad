\learningobjective{At the end of this challenge, the scholar will understand the concepts of users, groups, and permissions in a Linux system.}

\begin{challenge}
    \chatitle{Users and Groups in *NIX Systems}

    \begin{chadescription}
        Modern operating systems, such as Linux, provide a user-friendly interface for managing users, groups, and file permissions.
        When regular computers were still very expensive and filling a complete room, multiple Teletype Terminals were connected to the same machine at once.
        Thus, operating systems allowed for multiple users to share the same computer, each with their own settings, files, and security boundaries.
        These mechanisms are still in place, but with the addition of groups, which allow for more fine-grained control over permissions.
        For example:
        \begin{itemize}
            \item **Users**: Individual accounts, such as \texttt{alice}, \texttt{bob}, or \texttt{root}.
            \item **Groups**: Collections of users with shared access rights, such as \texttt{developers} or \texttt{admins}.
        \end{itemize}
        Each user is identified by a unique username and user ID (UID). 
        Additionally, users can belong to one or more groups, which help manage permissions for shared resources.
        A special user called the \texttt{root} user has full access to the system, while other users have limited permissions based on their group membership.
        You could also refer to root as the \textit{superuser} or \textit{administrator}.
        Since it is not always wanted to give every user full access to the root-account on a system, the root-account is protected by a password and that password is not shared among users.
        In order for a user to still have access to some adminstrative privileges, the user can use the \texttt{sudo} command.
        This is a special command that allows a user to temporarily gain the privileges of an administrator.
        Which privileges are granted to a user by the \texttt{sudo} command can be configured in a config-file called \texttt{sudoers}.\\
        When a file is accessed, the operating system checks if the user has the necessary permissions to access the file.
        These permissions are part of the file's metadata, which is stored in the file system.
        Permissions for files and directories are assigned based on three categories:
        \begin{enumerate}
            \item The **owner** of the file.
            \item The **group** associated with the file.
            \item **Others**, meaning everyone else.
        \end{enumerate}

        We will also briefly introduce the \texttt{root} user, the superuser with unrestricted privileges, and how other users can temporarily gain administrative rights using the \texttt{sudo} command.

        In this challenge, you will:
        \begin{enumerate}
            \item Learn how to view the current user and groups using \texttt{id}.
            \item Understand file ownership and permissions with \texttt{ls -l}.
            \item Explore why \texttt{sudo} is necessary for administrative tasks.
        \end{enumerate}
    \end{chadescription}

    \begin{task}
    Identify your user and groups:
    \begin{enumerate}
        \item Open a terminal.
        \item Run the command \texttt{id} and note the output.
        \item Identify your username, primary group, and any supplementary groups.
        \item Run \texttt{whoami} to confirm your current user.
    \end{enumerate}
    \begin{questions}
        \item What does the \texttt{id} command display?
        \item What is the difference between a primary group and supplementary groups?
        \item Why might a user belong to multiple groups?
    \end{questions}
    \end{task}

    \begin{task}
    Explore file ownership and permissions:
    \begin{enumerate}
        \item Navigate to your home directory using \texttt{cd ~}.
        \item Run \texttt{ls -l} to list the files in your home directory with detailed information.
        \item Observe the owner, group, and permission settings for each file.
    \end{enumerate}
    \begin{questions}
        \item Who owns the files in your home directory?
        \item What is the group associated with the files, and why?
        \item What do the permissions (e.g., \texttt{-rw-r--r--}) mean?
    \end{questions}
    \end{task}

    \begin{task}
    Introducing the \texttt{sudo} command:
    \begin{enumerate}
        \item Run the command \texttt{ls /root}. You should see a "permission denied" message because \texttt{/root} belongs to the \texttt{root} user.
        \item Try running \texttt{sudo ls /root}. You will be prompted for your password.
        \item Observe the difference in output between the two commands.
    \end{enumerate}
    \begin{questions}
        \item Why were you denied access to \texttt{/root} initially?
        \item What does \texttt{sudo} do, and why is it needed in this context?
        \item Why is it important to restrict access to certain directories like \texttt{/root}?
    \end{questions}
    \end{task}

    \begin{advice}
        Use \texttt{sudo} responsibly. It grants temporary administrative privileges, so misuse can affect the entire system. Always double-check your commands when using \texttt{sudo}.
    \end{advice}
\end{challenge}
